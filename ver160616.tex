\documentclass{article}
\usepackage[pagebackref=false,colorlinks=true,linkcolor=blue,citecolor=magenta]{hyperref} 
\usepackage{zref-perpage}
\zmakeperpage{footnote}
\usepackage{graphicx}
\usepackage{xepersian}

% % % % % % % % % % % % % % % % % %
\makeatletter
\@ifundefined{Umathcode}{\let\Umathcode\XeTeXmathcode}{}
\@ifundefined{Umathchardef}{\let\Umathchardef\XeTeXmathchardef}{}
\renewcommand{\@makefntext}[1]{\parindent 1em
   \noindent\hbox to 1em{}% if you want to indent footnote text you can change the width of the hbox (e.g. \hbox to 2em{})
   \llap{\if@RTL\else\latinfont\fi\@thefnmark\space}#1}
\makeatother
% % % % % % % % % % % % % % % % % %

\settextfont[Scale=1.4]{XB Yas}
\defpersianfont\Bes[Scale=8]{Besmellah1}
\setlatintextfont[Scale=1]{DejaVu Sans}

\frenchspacing

\addtolength{\hoffset}{-1.25cm}
\addtolength{\textwidth}{2.5cm}

\addtolength{\voffset}{-1.25cm}
\addtolength{\textheight}{2.5cm}


\begin{document}\itemsep5mm \parskip2pt 

\thispagestyle{empty}

\vspace*{2cm}
\begin{center}
{\huge
\textbf{
طرح جشنوارهٔ روز آزادی نرم‌افزار تهران
}
\\[1cm]}
{\Large
بنیاد دانش آزاد
}
{\large
\\[1cm]
\baselineskip=1cm
%{\Huge \yekan
\includegraphics[width=8cm]{images/logo.png}
\\[.4cm]
%}
خرداد ۱۳۹۵
\\[.4cm]

}


{\large
مجوز انتشار
\\[.4cm]

\Large
\includegraphics[width=4cm]{images/cc-by-sa-license.png}

}

\end{center}


\newpage
\thispagestyle{empty}
\vspace*{\fill}
\begin{center}
\Bes
\Huge
t
\end{center}
\vspace*{\fill}


\newpage
\thispagestyle{empty}
\tableofcontents %‫هر جا که ظاهر شود باعث نمایش فهرست مطالب در همان نقطه میشود‬




\newpage
\vspace*{2cm}

\section{شناسنامهٔ رویداد}

\begin{flushright}
\textbf{عنوان جشنواره:} روز آزادی نرم‌افزار ۱۳۹۵ در تهران

\textbf{سطح برگزاری:} \lr{✓} ملی \lr{⚪} منطقه‌ای \lr{⚪} بین‌المللی

\textbf{نام برگزارکننده:} بنیاد دانش آزاد ایران

\textbf{مدت زمان:} یک روز

\textbf{تاریخ برگزاری:} 

\textbf{ساعت برگزاری:} 


\textbf{محل برگزاری:}


\textbf{نحوه تامین مالی:} از محل

\begin{itemize}
\item[] \lr{✓} حمایت مردمی
\item[] \lr{✓} جذب حامیان مالی (اسپانسر)
\item[] \lr{✓} برگزاری نمایشگاه‌های جانبی
\item[] \lr{⚪} بودجه تایین‌شده از قبل
\item[] \lr{⚪} ثبت‌نام در جشنواره
\end{itemize}

\textbf{سازمان‌های همکار و حامیان اصلی تا این تاریخ:}

\begin{center}
    \begin{tabular}{ | p{6cm} | p{9cm} |}
    \hline
    نام سازمان & نوع حمایت/همکاری توافق‌شده \\ \hline
    سازمان فناوری اطلاعات (\lr{itc.ir}) & تامین سالن جشنوارهٔ سال ۱۳۹۲، 
    در اختیار قرار دادن مکان برگزاری جلسات هفتگی بنیاد دانش آزاد ایران از تاریخ ۸ اردیبهشت ۱۳۹۲ تا تاریخ ۸ تیر ۱۳۹۳ \\ \hline
    مرکز ملی توسعه و بکارگیری نرم‌افزارهای بومی و آزاد/متن‌باز ایران & پذیرایی جشنوارهٔ سال ۱۳۹۲ \\ \hline
    دانشگاه صنعتی شریف & تامین سالن جشنوارهٔ سال ۱۳۹۴ \\ \hline
    انجمن علمی کامپیوتر دانشگاه شریف & همکاری در برگزاری جشنوارهٔ سال ۱۳۹۴ \\ \hline
    شرکت بندار سوبان سامانه & در اختیار قرار دادن مکان جلسات داوری جشنوارهٔ سال ۱۳۹۴ و مکان جلسات بنیاد دانش آزاد تا کنون \\ \hline
    فناوران آنیسا & حمایت مالی از جشنوارهٔ سال ۱۳۹۴ \\ \hline
    فناوران آنیسا & حمایت مالی از جشنوارهٔ سال ۱۳۹۴ \\ \hline
    \end{tabular}
\end{center}

\textbf{محورهای اصلی جشنواره:} معرفی، ترویج، کاربرد و استفاده، جذب مشارکت‌کننده و حمایت جامعه از پروژه‌ها و تولید نرم‌افزار آزاد، دانش آزاد و سخت‌افزار آزاد

\textbf{مخاطبین جشنواره:} کاربران انجمن‌های آزاد مجازی و غیر مجازی، دانشجویان، دانش‌آموزان سال‌های آخر دبیرستان، کارمندان و مدیران سازمان‌ها و شرکت‌های دولتی و خصوصی، کارمندان و مدیران شرکت‌های فعال در حوزه تولید نرم‌افزار

\textbf{برنامه‌های جانبی جشنواره:} \lr{✓} بخش پوستر \lr{✓} کارگاه‌های آموزشی

\textbf{برآورد تعداد شرکت‌کنندگان:} بیش از ۸۰۰ نفر

\textbf{تعرفه‌های ثبت‌نام در کنفرانس، کارگاه‌ها و سایر موارد:} رایگان و بدون نیاز به ثبت‌نام

\textbf{نشانی وب‌سایت جشنواره:‌} \lr{\href{http://sfd.fsug.ir}{sfd.fsug.ir}}

\textbf{اطلاعات مربوط به دوره‌های قبل:}

\begin{center}
    \begin{tabular}{ | p{2cm} | p{3cm} | p{3cm} | p{5cm} |}
    \hline
    برگزارکننده & تاریخ برگزاری & محل برگزاری & تصاویر جشنواره \\ \hline
    بنیاد دانش آزاد ایران & چهارشنبه ۳ مهر ۱۳۹۲ ساعت ۸ - ۱۷ & فرهنگسرای هنر (ارسباران) & \lr{\href{http://sfd.fsug.ir/1392/photos}{sfd.fsug.ir/1392/photos}} \\ \hline
    بنیاد دانش آزاد ایران & پنج‌شنبه ۹ مهر ۱۳۹۴ ساعت ۹ - ۱۷ & سالن جابر دانشگاه صنعتی شریف و دانشکدهٔ کامپیوتر & \lr{\href{}{sfd.fsug.ir/1392/photos}} \\ \hline
    \end{tabular}
\end{center}

برای اطلاع از رویدادهای مشابه داخلی یا خارجی (کنفرانس، نمایشگاه و ...) که به مناسبت روز آزادی نرم‌افزار آزاد تاکنون برگزارشده یا در آینده برگزار خواهند شد به عنوان \underline{۴. تاریخچهٔ روز آزادی نرم‌افزار} مراجعه کنید.

فهرست مهم‌ترین فعالیت‌هایی که تاکنون در بخش‌های علمی و اجرائی جشنواره انجام گرفته
در بورد ترللوی جشنواره به نشانی \lr{\href{http://sfd.fsug.ir/}{sfd.fsug.ir/}} موجود است.

خدمات مورد انتظار ما و روش‌های حمایت ذیل عنوان \underline{۵. اهداف کلی همایش} و \underline{۶. منافع حمایت از جامعه} به تفصیل مورد بررسی قرار گرفته است. درصورت تمایل برای همکاری/حمایت از این رویداد با تلفن همراه یا ایمیل شخص رابط تماس حاصل فرمایید.

\textbf{نام شخص رابط:} امیرحسین گودرزی

\textbf{تلفن همراه شخص رابط:} ۰۹۱۹۷۸۱۲۵۸۸

\textbf{ایمیل شخص رابط:} \lr{\href{mailto:sfd@fsug.ir}{sfd@fsug.ir}}

\end{flushright}

\section{نرم‌افزار آزاد چیست؟}
\subsection{تاریخچه}
نرم‌افزار آزاد زمانی به وجود آمد که ذهن خلاق فردی به نام ریچارد استالمن\LTRfootnote{Richard Mattew Stallman} تحمل اشکالات نرم‌افزارهای انحصاری آن زمان مانند یونیکس\LTRfootnote{Unix} را نداشت. او دوست نداشت برای استفاده از هر نرم‌افزاری اجازه بگیرد یا در استفاده از نرم‌افزارها برای انجام کار‌های دلخواهش محدود شود، لذا تصمیم گرفت پروژه‌ای را راه‌اندازی کند که به این انحصار پایان دهد. در سال ۱۹۸۳ پروژهٔ GNU را در آزمایشگاه هوش مصنوعی دانشگاه MIT آغاز کرد و در سال ۱۹۸۵ بنیاد نرم‌افزار‌های آزاد\LTRfootnote{Free Software Foundation} را بنا نهاد.
\subsection{تعریف نرم‌افزار آزاد}
نرم‌افزار آزاد، نرم‌افزاری است که به‌همراه کد منبع توزیع شده و با قوانینی منتشر می‌شود که آزادی استفاده، بررسی، ویرایش، بهبود و بازنشر آن را تضمین می‌کند. لذا نرم‌افزاری آزاد نامیده می‌شود که این آزادی‌ها را داشته باشد.
\subsection{خرید و فروش و کسب درآمد از طریق نرم‌افزار آزاد}

در نرم‌افزارهای انحصاری مانند محصولات شرکت مایکروسافت\LTRfootnote{Microsoft} و اپل\LTRfootnote{Apple} شما برای داشتن نرم‌افزار یا باید از راه غیر‌قانونی اقدام به تهیه آن نرم‌افزار کنید یا این‌که با پرداخت هزینه‌ای هنگفت از شرکت مربوطه مجوزی خریداری نمایید که به شما حق استفاده از آن نرم‌افزار را بدهند، اما نرم‌افزار آزاد اینگونه نیست! شما برای استفاده از نرم‌افزار آزاد مجبور به خرید مجوز نیستید. اما در جواب اینکه چطور می‌توان نرم‌افزار آزاد را فروخت، باید گفت که می‌توان از طریق فروش فایل دودویی\LTRfootnote{Binary} یا راهنما‌ها و یا فروش خدمات راه‌اندازی سرویس، ارائه خدمات و یا حتی فروش نسخه‌های تغییر‌یافته نرم‌افزاری خاص بدون فروش مجوز، کسب در‌آمد کرد.
\section{معرفی جوامع کاربری نرم‌افزار آزاد و دانش آزاد}
گروه‌های بسیاری در ایران و بقیهٔ کشورهای جهان بصورت فیزیکی یا آنلاین در زمینهٔ نرم‌افزار آزاد، سیستم‌عامل‌ گنو/لینوکس، دانش آزاد، آموزش و نشر آزاد دوره‌های علمی، نشر مقالات آزاد علمی و ... فعالیت می‌کنند. فعالیت این گروه‌ها در بیشتر موارد توسط افراد داوطلب و به صورت خودجوش انجام می‌گیرد.
به عنوان مثال بنیاد نرم‌افزار آزاد سی سال است که در راستای گسترش نرم‌افزار آزاد، دانش آزاد کامپیوتر و حفظ حریم شخصی افراد فعالیت‌های مستمری انجام می‌دهد و در نتیجهٔ این فعالیت‌ها توانسته است سیستم‌عامل آزاد گنو/لینوکس را به جایگاه شایسته‌ای در جهان تکنولوژی امروز برساند. نرم‌افزار آزاد در هر زمینه‌ای از دنیای دیجیتال که وارد شده است اگر نگوییم که گوی سبقت را از رقیبان انحصاری خود ربوده است، پابه‌پای آنان در دنیای مدرن امروز به پیش رفته است.
بنیاد ویکی‌پدیا که حاصل تزویج نرم‌افزار آزاد و دانش آزاد است امروزه به بزرگترین مرجع دانش در سراسر دنیا بدل گشته و نه‌تنها بزرگترین دانشنامهٔ عمومی این قرن است بلکه با هزینه‌ی کمک‌های داوطلبانه‌ی افراد علاقه‌مند به گسترش دانش آزاد اداره می‌شود.
در ایران جوامع نرم‌افزار آزاد عمر طولانی‌تری نسبت به جوامع دانش‌آزاد دارند. گروه کاربران لینوکس تهران (تهران‌لاگ) یکی از اولین جوامع فعال نرم‌افزار آزاد در ایران است. فعالیت لاگ‌ها اغلب شامل برقراری جلسات حضوری با ارائه‌های تخصصی و نیمه تخصصی مرتبط با فناوری‌های آزاد و لینوکس، برگزاری همایش‌ها، جشن‌های انتشار و ... می‌شود.
گروه دیگر وجود گروه‌های کاربران نرم‌افزار‌های آزاد است که فعالیت‌های آن بیشتر به جای ارائه‌های تخصصی یک‌طرفه، حول محور آشنایی تازه‌کاران با نرم‌افزار آزاد، بحث و تبادل دانش بین بقیه افراد گروه، توسعه گروهی نرم‌افزار‌های مورد نیاز جامعه، برگزاری کارگاه‌های آموزشی، همایش‌ روز آزادی نرم‌افزار و … می‌گردد. دغدغه این گروه بیشتر آشنایی افراد ناآشنا با مفاهیم نرم‌افزار آزاد است.
دستهٔ دیگر گروه‌های فعال در زمینه دانش آزاد است. از دیرباز دانش‌آزاد مورد توجه دانش دوستان بوده و هست. لذا گروه‌های بسیاری به‌صورت مجازی و فیزیکی از گذشته مشغول به فعالیت هستند که برای دستهٔ مجازی آن می‌توان به سایت کلاس درس\LTRfootnote{\lr{\href{http://kelasedars.org}{kelasedars.org}}}
 که محلی‌ست برای آموزش دروس دبیرستان به صورت آزاد، مکتب‌خونه\LTRfootnote{\lr{\href{http://maktabkhooneh.org}{maktabkhooneh.org}}} محلی برای نشر دانش آزاد دانشگاهی،‌ مرکز گسترش عدالت آموزشی (آلا)\LTRfootnote{\lr{\href{http://sanatisharif.ir}{sanatisharif.ir}}} و برای دسته فعالیت‌های فیزیکی این گروه‌ها به بنیاد دانش‌آزاد\LTRfootnote{\lr{\href{http://www.fsug.ir/wiki}{www.fsug.ir/wiki}}} اشاره کرد که در زمینه‌های کامپیوتر، برق، علوم انسانی و ... فعالیت می‌کند.
\section{تاریخچهٔ روز آزادی نرم‌افزار}
کاربران نرم‌افزارهای آزاد روز آزادی نرم‌افزار\LTRfootnote{Software Freedom Day} را جشن می‌گیرند. هدف اصلی این جشن‌ها، معرفی این مفهوم به دیگران و تشویق آن‌ها به استفاده از نرم‌افزارهای آزاد است.
اولین بار این روز در ۲۸ اوت سال ۲۰۰۴ توسط ۷۰ گروه در نقاط مختلف جهان جشن گرفته شد. هر ساله نزدیک ۳۰۰ گروه کاربری در سراسر دنیا این روز را جشن می‌گیرند. سال‌های پیش به‌جز تهران این جشن توسط گروه‌های فعال کاربری شاهین‌شهر، اصفهان، کرج و دانشگاه علم و صنعت برگزار شد. سال ۱۳۹۲بنیاد دانش آزاد مجری برگزاری این جشن در تهران بود.
\subsection{روز آزادی نرم‌افزار سال ۱۳۹۲ در تهران}
جشنوارهٔ روز آزادی نرم‌افزار سال ۱۳۹۲ در تهران\LTRfootnote{\lr{\href{http://sfd.fsug.ir/1392}{sfd.fsug.ir/1392}}} با حمایت سازمان فناوری اطلاعات و سازمان فرهنگی‌هنری شهرداری تهران و مرکز توسعه و بکارگیری نرم‌افزارهای آزاد/متن‌باز و با محوریت آشناسازی جامعه با نرم‌افزار آزاد و بررسی تفاوت‌های آن با سایر سیستم‌های موجود، در تاریخ چهارشنبه ۳ مهر ۱۳۹۲ ساعت ۸ الی ۱۶:۳۰، در دو بخش کارگاه‌های تخصصی و ارائه‌ها برگزار شد. به‌صورت همزمان این جشن در تاریخ ۳۰ شهریور ۱۳۹۲  در بسیاری از نقاط دنیا با تمرکز بر نرم‌افزارهای آزاد برگزار شد. لازم به ذکر است که شرکت در این جشنواره برای عموم علاقه‌مندان آزاد و رایگان می‌باشد.

\subsection{روز آزادی نرم‌افزار سال ۱۳۹۴ در تهران}
جشنوارهٔ روز آزادی نرم‌افزار سال ۱۳۹۴ در تهران\LTRfootnote{\lr{\href{http://sfd.fsug.ir/1394}{sfd.fsug.ir/1394}}} با همکاری انجمن علمی دانشکدهٔ کامپیوتر دانشگاه شریف، در تاریخ پنج‌شنبه ۹ مهر ۱۳۹۴ ساعت ۹ الی ۱۷، در دو بخش کنفرانس و کارگاه‌های آموزشی برگزار شد. در طول این مدت بیشتر از ۸۰۰ نفر از شهرهای مختلف ایران در بخش‌های کنفرانس و کارگاه‌ها شرکت کردند. شرکت در این جشنواره برای همهٔ علاقه‌مندان، آزاد و رایگان بود.

بیشتر سخنرانی‌های بخش کنفرانس با موضوعات غیرفنی بود. ارائه‌دهندگان بخش کنفرانس دربارهٔ موضوعات اهمیت حریم شخصی، تاکید بر اهمیت جامعهٔ نرم‌افزار آزاد، توضیحاتی در مورد نرم‌افزار آزاد، معرفی مدارک LPIC، آشنایی با درس حجیم آنلاین آزاد (ماک)، نرم‌افزار آزاد در کلان‌داده، «چیزهایی دربارهٔ ویکی‌پدیا که نمی‌دانستید»، معرفی بنیاد نرم‌افزارهای آزاد/متن‌باز ایران و سامانهٔ DLP سخنرانی کردند. ابتدای مراسم فیلم پیام ریچارد استالمن که به مناسبت جشنوارهٔ روز آزادی نرم‌افزار تهران ضبط کرده بود، پخش شد. همچنین در بخش‌‌های استراحت، دو انیمیشن آزاد بنیاد بلندر به نام‌های Sintel و Cosmos Laundromat، مصاحبهٔ نلسون ماندلا دربارهٔ اوبونتو و کلیپ We are Linux پخش شد.

در بخش کارگاه‌ها، کارگاه‌های متنوعی با موضوعات فنی و عمومی به صورت موازی با بخش کنفرانس برگزار شد. کارگاه نصب و استفاده از سیستم عامل گنو/لینوکس در لابی دانشکدهٔ کامپیوتر دانشگاه صنعتی شریف از ساعت ۱۰ تا ۱۶ برگزار شد. سایر کارگاه‌ها با موضوعات ابزارهای توسعهٔ ویکی‌پدیا، معرفی ابزارهای حفظ حریم شخصی، آموزش نرم‌افزار بلندر، آموزش فریم‌ورک‌های فرانت‌اند، سخت‌افزار آزاد و آردوئینو،  کارگاه Embedded Linux From Scratch، سه کارگاه امنیت (با موضوعات SELinux، کالی لینوکس، هک کلاه سفید و تست نفوذ به شبکه)،  دو کارگاه برنامه‌نویسی (با موضوعات نوشتن اپلیکشن با Qt و آشنایی زبان برنامه‌نویسی R)، و دو کارگاه برنامه‌نویسی وب (با موضوعات آموزش فریمورک YII و کامپوننت‌نویسی برای جوملا) و سامانهٔ کنترل ورژن گیت در کلاس‌های دانشکدهٔ کامپیوتر برگزار شد.

همچنین یکی از بخش‌های موازی در این جشنواره که در لابی دانشکدهٔ کامپیوتر برگزار شد، بخش نقاشی لوگوهای نرم‌افزار آزاد روی تیشرت بود. در این بخش افراد تیشرت‌های خود را در اختیار برگزارکنندهٔ این بخش قرار دادند و ایشان روی لباس‌های افراد، طرح‌های مورد نظر را نقاشی می‌کرد.

در این جشنواره ۱۶ نفر به عنوان تیم اجرایی و برگزارکنندهٔ رویداد و ۱۹ نفر به عنوان استف در روز جشنواره مشارکت کردند و از میان ۳۳ نفری که در سایت جشنواره مقاله یا پیشنهاد برگزاری کارگاه ثبت کردند بعد از برگزاری جلسات داوری ۱۳ نفر در بخش کارگاه‌ها و ۱۰ نفر در بخش همایش مطالب خود را ارائه کردند و شش عدد کلیپ جمعا به مدت ۴۸ دقیقه در سالن اصلی پخش شد. از مجموع ۲۴ ساعت محتوای ارائه شده در بخش کارگاه‌ها و همایش، بعد از تدوین جمعا ۱۲ ساعت فیلم سخنرانی به دست آمد که هر کدام در پنج قالب مختلف در سایت‌های یوتیوب، تخته‌سفید و آرشیو آپلود شدند. مجموع ویدئوهای جشنوارهٔ روز آزادی نرم‌افزار سال ۱۳۹۴ در کمتر از یک سال بیشتر از ۵۰۰۰ بار در سایت‌های مختلف بازدید داشته‌اند.

بازتاب جشنواره در رسانه‌ها به گونه‌ای بود که تعدادی از نشریات، فروم‌های اینترنتی، بلاگ‌ها، سایت‌های خبری، سایت‌های مرتبط با نرم‌افزار آزاد و سایت‌های شرکت‌های فناوری اطلاعات، خبر برگزاری جشنواره را منتشر کردند. همچنین شبکهٔ تلویزیونی الکوثر گزارشی به زبان عربی از جشنواره تهیه و پخش کرد.

\section{اهداف کلی همایش}
سعی بر آن شده تا با برگزاری ارائه‌ها و کارگاه‌های متنوع و با بهره‌گیری از تجارب افراد فعال در جامعهٔ نرم‌افزار آزاد و دانش آزاد، قدم‌هایی در راستای گسترش این مفاهیم در جامعه برداشته شود. به طور خلاصه می‌توان گفت محورهای اصلی برنامه‌های این جشنواره معرفی، ترویج، افزایش کاربرد و استفاده، جذب مشارکت افراد و جلب حمایت جامعه و تولید در زمینه‌ی نرم‌افزار آزاد، دانش آزاد و سخت‌افزار/روباتیک آزاد است. به طور کلی اهداف همایش به دو صورت دسته‌بندی می‌شود.

\subsection{اهداف موضوعی}
\begin{flushright}

\begin{itemize}
\item نرم‌افزار آزاد
\item سخت‌افزار/رباتیک آزاد
\item دانش آزاد
\end{itemize}
\end{flushright}

\subsection{اهداف اجرایی}

\begin{flushright}
\begin{itemize}
\item ترویج
\item کاربرد و استفاده
\item معرفی
\item جذب مشارکت‌کننده و حمایت جامعه
\item تولید
\end{itemize}
\end{flushright}

به ازای هر کدام از اهداف موضوعی پنج هدف اجرایی تعریف شده‌است. برای نمونه نرم‌افزار آزاد در اجرا پنج ما به‌ازا دارد: ترویج نرم‌افزار آزاد، کاربرد و استفاده از نرم‌افزار آزاد، معرفی نرم‌افزار آزاد، جذب مشارکت‌کننده و حمایت جامعه از نرم‌افزار آزاد و پروژه‌های مرتبط و تولید نرم‌افزار آزاد. برای بقیه اهداف موضوعی مانند دانش آزاد و سخت‌افزار/روباتیک آزاد نیز اهداف اجرایی به این صورت تعیین شده‌است.

\section{منافع حمایت از جامعه}
مخاطبان این رویداد از هر گروه و دسته‌ای مانند توسعه‌دهندگان نرم‌افزار، مدیران شبکه‌های کامپیوتری، دانشجویان، مدیران صنایع و افراد علاقه‌مند به حوزهٔ IT  هستند.

\begin{flushright}

به طور کلی جشنوارهٔ روز‌ آزادی نرم‌افزار دو دسته مخاطب دارد:
\begin{itemize}
\item افراد فعال و نخبه در زمینهٔ نرم‌افزار آزاد، شبکه، امنیت و اینترنت
\item افراد علاقه‌مند به نرم‌افزار، کامپیوتر، اینترنت و البته آزادی.
\end{itemize}

\end{flushright}
\begin{flushright}

باتوجه به ماهیت نرم‌افزار آزاد حضور شما به عنوان حامی در این رویداد می‌تواند علاوه‌ بر برد تبلیغاتی برند شما، یک حرکت اخلاقی و معنوی در راستای گسترش سطح دانش و رفاه جامعه باشد. به طور کلی نتایجی که حمایت از جامعه در پی دارد به شرح زیر است:
\begin{itemize}
\item حمایت از یک حرکت اخلاقی و معنوی
\item گسترش سطح دانش و رفاه جامعه
\item معرفی فعالیت‌ها، تاریخچه و برند شرکت
\item شناسایی افراد فعال و نخبه و جذب آن‌ها
\item آشنایی و ارتباط با اعضای بقیه گروه‌های کاربری فعال در حوزه‌ی نرم‌افزار ازاد و دانش آزاد
\item ارتباط با مدیران و صاحبان صنایع مرتبط با نرم‌افزار آزاد/متن‌باز و افراد فعال و دارای دانش کافی در این زمینه
\end{itemize}

\end{flushright}

\subsection{مواردی که می‌توانید به عهده بگیرید}


\begin{flushright}

\begin{itemize}
\item ناهار
\item پذیرایی
\item اینترنت
\item پخش زنده
\item تبلیغات کاغذی
\end{itemize}

\end{flushright}

\subsection{مواردی که می‌توانیم به عهده بگیریم}

\subsubsection{امتیازات ویژه}

\begin{flushright}
\begin{itemize}
\item قرارگیری و توزیع بروشور در بسته برنامه
\item قرارگیری فرم جذب نیروی کار در سایت برای مدت معلوم
\item اختصاص یک عدد میز برای توزیع بروشور، فرم جذب نیروی کار و معرفی شفاهی شرکت در فضای پذیرایی بیرون سالن
\item درج نشان شرکت روی پوسترهای جشنواره و بنرهای اینترنتی
\item قرارگیری پرچم حامی در سن برنامه
\item اهدای تندیس و لوح تقدیر از حامی از طرف دبیرخانه جشنواره
\item تقدیر از حامی در پایان برگزاری مراسم
\end{itemize}
\end{flushright}

\subsubsection{امتیازات معمولی}
\begin{flushright}


\begin{itemize}
\item ثبت نام و نشانه و درج لینک شرکت در بخش حامیان ویژه در سایت جشنواره
\item نصب استند در بیرون سالن
\item قرار گیری و توزیع فرم جذب نیروی کار در بسته برنامه
\item درج نام شرکت در تاریخچهٔ سایت برای همیشه
\item معرفی حامی در شبکه‌های اجتماعی مربوط به رویداد
\item درج لوگو و نام حامی در ایمیل‌های گروهی
\item معرفی حامی در اطلاع‌رسانی‌ها و اخبار ارائه‌شده توسط حامیان رسانه‌ای
\end{itemize}
\end{flushright}


\end{document}



