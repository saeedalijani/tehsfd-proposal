\documentclass{article}
\usepackage[pagebackref=false,colorlinks=true,linkcolor=blue,citecolor=magenta]{hyperref} 
\usepackage{zref-perpage}
\zmakeperpage{footnote}
\usepackage{graphicx}
\usepackage{xepersian}

% % % % % % % % % % % % % % % % % %
\makeatletter
\@ifundefined{Umathcode}{\let\Umathcode\XeTeXmathcode}{}
\@ifundefined{Umathchardef}{\let\Umathchardef\XeTeXmathchardef}{}
\renewcommand{\@makefntext}[1]{\parindent 1em
   \noindent\hbox to 1em{}% if you want to indent footnote text you can change the width of the hbox (e.g. \hbox to 2em{})
   \llap{\if@RTL\else\latinfont\fi\@thefnmark\space}#1}
\makeatother
% % % % % % % % % % % % % % % % % %

\settextfont[Scale=1.4]{XB Yas}
\defpersianfont\Bes[Scale=8]{Besmellah1}
\setlatintextfont[Scale=1]{DejaVu Sans}

\frenchspacing

\addtolength{\hoffset}{-1.25cm}
\addtolength{\textwidth}{2.5cm}

\addtolength{\voffset}{-1.25cm}
\addtolength{\textheight}{2.5cm}


\begin{document}\itemsep5mm \parskip2pt 

\thispagestyle{empty}

\vspace*{2cm}
\begin{center}
{\huge
\textbf{
طرح جشنوارهٔ روز آزادی نرم‌افزار تهران
}
\\[1cm]}
{\Large
بنیاد دانش آزاد
}
{\large
\\[1cm]
\baselineskip=1cm
%{\Huge \yekan
\includegraphics[width=8cm]{images/logo.png}
\\[.4cm]
%}
تیر ۱۳۹۵
\\[.4cm]

}


{\large
مجوز انتشار
\\[.4cm]

\Large
\includegraphics[width=4cm]{images/cc-by-sa-license.png}

}

\end{center}


\newpage
\thispagestyle{empty}
\vspace*{\fill}
\begin{center}
\Bes
\Huge
t
\end{center}
\vspace*{\fill}


\newpage
\thispagestyle{empty}
\tableofcontents %‫هر جا که ظاهر شود باعث نمایش فهرست مطالب در همان نقطه میشود‬




\newpage
\vspace*{2cm}

\section{شناسنامهٔ رویداد}

\begin{flushright}
\textbf{عنوان جشنواره:} روز آزادی نرم‌افزار ۱۳۹۵ در تهران

\textbf{سطح برگزاری:} \lr{✓} ملی \lr{⚪} منطقه‌ای \lr{⚪} بین‌المللی

\textbf{برگزارکننده:} بنیاد دانش آزاد

\textbf{نشانی وب‌سایت جشنواره:‌} \lr{\href{http://sfd.fsug.ir}{sfd.fsug.ir}}

\textbf{مدت زمان:} یک روز

\textbf{تاریخ مورد نظر:} پنج‌شنبه یا جمعه هفتهٔ اول مهر ۱۳۹۵

\textbf{ساعت برگزاری:} از ۸ تا ۱۷

\textbf{محل برگزاری مورد نیاز:} سالن کنفرانس برای بخش ارائه‌ها، چند کلاس برای برگزاری کارگاه‌های آموزشی به صورت موازی و مکان تبلیغات اسپانسرها

\textbf{برآورد تعداد شرکت‌کنندگان:} بیش از ۸۰۰ نفر

\textbf{محورهای اصلی جشنواره:} معرفی، ترویج، کاربرد و استفاده، جذب مشارکت‌کننده و حمایت جامعه از پروژه‌ها و تولید نرم‌افزار آزاد، دانش آزاد و سخت‌افزار آزاد

\textbf{مخاطبین جشنواره:} کاربران انجمن‌های نرم‌افزاری، دانشجویان، دانش‌آموزان سال‌های آخر دبیرستان، کارمندان و مدیران سازمان‌ها و شرکت‌های دولتی و خصوصی، کارمندان و مدیران شرکت‌های فعال در حوزهٔ تولید نرم‌افزار

\textbf{برنامه‌های جانبی جشنواره:} \lr{✓} بخش پوستر \lr{✓} کارگاه‌های آموزشی

\textbf{تعرفه‌های ثبت‌نام در کنفرانس، کارگاه‌ها و سایر موارد:} رایگان و بدون نیاز به ثبت‌نام

\textbf{نحوهٔ تامین مالی:} از محل

\begin{itemize}
\item[] \lr{✓} حمایت مردمی
\item[] \lr{✓} جذب حامیان مالی (اسپانسر)
\item[] \lr{✓} برگزاری نمایشگاه‌های جانبی
\item[] \lr{✓} بودجه تایین‌شده از قبل
\item[] \lr{⚪} ثبت‌نام در جشنواره
\end{itemize}
\clearpage

\textbf{اطلاعات مربوط به دوره‌های قبل:}

\begin{center}
	\def\arraystretch{2}
    \begin{tabular}{ | p{2.5cm} | p{3.6cm} | p{3cm} | p{4cm} |}
    \hline
    \textbf{برگزارکننده} & \textbf{تاریخ برگزاری} & \textbf{محل برگزاری} & \textbf{تصاویر جشنواره} \\ \hline
\hline
    بنیاد دانش آزاد & چهارشنبه ۳ مهر ۱۳۹۲ ساعت ۸ - ۱۷ & فرهنگسرای هنر (ارسباران) & \lr{\href{http://sfd.fsug.ir/1392/photos}{sfd.fsug.ir/1392/photos}} \\ \hline
    بنیاد دانش آزاد & پنج‌شنبه ۹ مهر ۱۳۹۴ ساعت ۹ - ۱۷ & سالن جابر دانشگاه صنعتی شریف و دانشکدهٔ کامپیوتر & \lr{\href{http://sfd.fsug.ir/1394/photos}{sfd.fsug.ir/1394/photos}} \\ \hline
    \end{tabular}
\end{center}

\vspace*{1cm}

\textbf{سازمان‌های همکار و حامیان اصلی تا این تاریخ:}

\begin{center}
	\def\arraystretch{2}
    \begin{tabular}{ | p{6cm} | p{7.75cm} |}
    \hline
    \textbf{نام سازمان} & \textbf{نوع حمایت/همکاری توافق‌شده} \\ \hline
\hline
    سازمان فناوری اطلاعات (\lr{itc.ir}) & تامین سالن فرهنگسرای ارسباران برای جشنوارهٔ سال ۱۳۹۲، 
    در اختیار قرار دادن مکان برگزاری جلسات هفتگی بنیاد دانش آزاد ایران از تاریخ ۸ اردیبهشت ۱۳۹۲ تا تاریخ ۸ تیر ۱۳۹۳ \\ \hline
    مرکز ملی توسعه و بکارگیری نرم‌افزارهای بومی و آزاد/متن‌باز ایران & پذیرایی جشنوارهٔ سال ۱۳۹۲ و حمایت مالی از جشنوارهٔ سال ۱۳۹۴\\ \hline
    دانشگاه صنعتی شریف & تامین سالن جابرابن‌حیان برای جشنوارهٔ سال ۱۳۹۴ \\ \hline
    انجمن علمی کامپیوتر دانشگاه شریف & همکاری در برگزاری جشنوارهٔ سال ۱۳۹۴ \\ \hline
    شرکت بندار سوبان سامانه & در اختیار قرار دادن مکان جلسات داوری جشنوارهٔ سال ۱۳۹۴ و مکان جلسات بنیاد دانش آزاد از ۲۰ مرداد ۱۳۹۳ تا کنون \\ \hline
    فناوران آنیسا & حمایت مالی از جشنوارهٔ سال ۱۳۹۴ \\ \hline
    بستنی و لبنیات میهن & پذیرایی شرکت‌کنندگان از ساعت ۹ تا ۱۳ در جشنوارهٔ سال ۱۳۹۴ \\ \hline
    پروژهٔ تور &  	ارسال پکیج تیشرت و برچسب با طرح تور از آلمان برای جشنوارهٔ سال ۱۳۹۴ \\ \hline
    شرکت پارس‌پک & دادن هدیه‌های ارائه‌های برتر جشنوارهٔ سال ۱۳۹۴\\ \hline
    \end{tabular}
\end{center}

برای اطلاع از رویدادهای مشابه داخلی یا خارجی (کنفرانس، نمایشگاه و ...) که به مناسبت روز آزادی نرم‌افزار آزاد تاکنون برگزارشده یا در آینده برگزار خواهند شد به عنوان \underline{۴. تاریخچهٔ روز آزادی نرم‌افزار} مراجعه کنید.

خدمات مورد انتظار ما و روش‌های حمایت ذیل عنوان \underline{۵. اهداف کلی همایش} و \underline{۶. منافع حمایت از جامعه} به تفصیل مورد بررسی قرار گرفته است. درصورت تمایل برای همکاری/حمایت از این رویداد با تلفن همراه یا ایمیل شخص رابط تماس حاصل فرمایید.

\textbf{نام شخص رابط:} سعید علیجانی

\textbf{تلفن همراه شخص رابط:} ۰۹۱۹۵۶۱۶۰۵۹

\textbf{ایمیل شخص رابط:} \lr{\href{mailto:sfd@fsug.ir}{sfd@fsug.ir}}

\end{flushright}

\section{نرم‌افزار آزاد چیست؟}
\subsection{تاریخچه}
نرم‌افزار آزاد زمانی به وجود آمد که ذهن خلاق فردی به نام ریچارد استالمن\LTRfootnote{Richard Mattew Stallman} تحمل اشکالات نرم‌افزارهای انحصاری آن زمان مانند یونیکس\LTRfootnote{Unix} را نداشت. او نخواست برای استفاده از هر نرم‌افزاری اجازه بگیرد یا در استفاده از نرم‌افزارها برای انجام کار‌های دلخواهش محدود شود، لذا تصمیم گرفت پروژه‌ای را راه‌اندازی کند که به این انحصار پایان دهد. در سال ۱۹۸۳ پروژهٔ \lr{GNU} را در آزمایشگاه هوش مصنوعی دانشگاه \lr{MIT} آغاز کرد و در سال ۱۹۸۵ بنیاد نرم‌افزار‌های آزاد\LTRfootnote{Free Software Foundation} را بنا نهاد.

\subsection{تعریف نرم‌افزار آزاد}
نرم‌افزار آزاد، همراه کد منبع و قوانینی منتشر می‌شود که آزادی استفاده، بررسی، ویرایش، بهبود و بازنشر کد نرم‌افزار را تضمین می‌کنند. نرم‌افزاری که این چهار آزادی را داشته باشد نرم‌افزار آزاد نامیده می‌شود. در نرم‌افزارهای غیرآزاد افراد برای استفاده از نرم‌افزار باید مجوز آن را خریداری کنند یا از راه غیر قانونی از آن استفاده کنند، اما در نرم‌افزار آزاد افراد برای استفاده مجبور به خرید مجوز نیستند. در نرم‌افزار آزاد از راه‌هایی مانند فروش فایل دودویی\LTRfootnote{Binary} یا راهنماها، فروش خدمات، راه‌اندازی سرویس می‌توان کسب درآمد کرد.

\section{معرفی جوامع کاربری نرم‌افزار آزاد و دانش آزاد}
گروه‌های بسیاری در ایران و بقیهٔ کشورهای جهان بصورت فیزیکی یا آنلاین در زمینهٔ نرم‌افزار آزاد، سیستم‌عامل‌ گنو/لینوکس، دانش آزاد، آموزش و نشر آزاد دوره‌های علمی، نشر مقالات آزاد علمی و ... فعالیت می‌کنند. فعالیت این گروه‌ها در بیشتر موارد توسط افراد داوطلب و به صورت خودجوش انجام می‌گیرد.
به عنوان مثال بنیاد نرم‌افزار آزاد سی سال است که در راستای گسترش نرم‌افزار آزاد، دانش آزاد کامپیوتر و حفظ حریم شخصی افراد فعالیت‌های مستمری انجام می‌دهد و در نتیجهٔ این فعالیت‌ها توانسته است سیستم‌عامل آزاد گنو/لینوکس را به جایگاه شایسته‌ای در جهان تکنولوژی امروز برساند. نرم‌افزار آزاد در هر زمینه‌ای از دنیای دیجیتال که وارد شده است اگر نگوییم که گوی سبقت را از رقیبان انحصاری خود ربوده است، پابه‌پای آنان در دنیای مدرن امروز به پیش رفته است.
ویکی‌پدیا که حاصل تزویج نرم‌افزار آزاد و دانش آزاد است امروزه نه‌تنها بزرگترین دانشنامهٔ عمومی این قرن است بلکه با هزینهٔ کمک‌های داوطلبانهٔ افراد علاقه‌مند به گسترش دانش آزاد اداره می‌شود. در ایران جوامع نرم‌افزار آزاد عمر طولانی‌تری نسبت به جوامع دانش‌آزاد دارند. گروه کاربران لینوکس تهران (تهران‌لاگ) یکی از اولین جوامع فعال نرم‌افزار آزاد در ایران است. دستهٔ دیگر گروه‌های فعال در زمینهٔ دانش آزاد هستند که به‌صورت مجازی و فیزیکی از گذشته مشغول به فعالیتند که برای دستهٔ مجازی آن می‌توان به سایت‌های کلاس درس\LTRfootnote{\lr{\href{http://kelasedars.org}{kelasedars.org}}}، مکتب‌خونه\LTRfootnote{\lr{\href{http://maktabkhooneh.org}{maktabkhooneh.org}}}،‌ مرکز گسترش عدالت آموزشی (آلا)\LTRfootnote{\lr{\href{http://sanatisharif.ir}{sanatisharif.ir}}} و برای دسته فعالیت‌های فیزیکی این گروه‌ها به بنیاد دانش‌آزاد\LTRfootnote{\lr{\href{http://www.fsug.ir/wiki}{www.fsug.ir/wiki}}} اشاره کرد که حدود شش سال است که در زمینه‌های کامپیوتر، برق، علوم انسانی و ... فعالیت می‌کند. هدف بنیاد دانش آزاد فعالیت عمیق‌تر در همهٔ جنبه‌‌های اجتماعی و بسط دادن عوامل گسترش نرم‌افزار آزاد و بررسی ریشه‌‌های ارتباطی جنبه‌های مختلف نرم‌افزار آزاد و اجتماع است.

\section{تاریخچهٔ روز آزادی نرم‌افزار}
کاربران نرم‌افزار آزاد روز آزادی نرم‌افزار\LTRfootnote{Software Freedom Day} را جشن می‌گیرند. هدف اصلی این رویدادها، معرفی و تشویق دیگران به استفاده از نرم‌افزارهای آزاد است.
اولین بار این روز در ۲۸ اوت سال ۲۰۰۴ توسط ۷۰ گروه در نقاط مختلف جهان جشن گرفته شد. هر ساله بیش از ۳۰۰ گروه کاربری در سراسر دنیا این رویداد را برگزار می‌کنند. سال‌های پیش به‌جز تهران این رویداد توسط گروه‌های فعال کاربری شاهین‌شهر، اصفهان، رشت، مشهد، کرج و دانشگاه علم و صنعت برگزار شد. سال ۱۳۹۲ و ۱۳۹۴بنیاد دانش آزاد مجری برگزاری این جشنواره در تهران بود.

\subsection{روز آزادی نرم‌افزار سال ۱۳۹۲ در تهران}
جشنوارهٔ روز آزادی نرم‌افزار سال ۱۳۹۲ در تهران\LTRfootnote{\lr{\href{http://sfd.fsug.ir/1392}{sfd.fsug.ir/1392}}} با محوریت آشناسازی جامعه با نرم‌افزار آزاد و بررسی تفاوت‌های آن با سایر سیستم‌های موجود، در تاریخ چهارشنبه ۳ مهر ۱۳۹۲ ساعت ۸ الی ۱۶:۳۰، در دو بخش کارگاه‌های تخصصی و ارائه‌ها در فرهنگسرای ارسباران برگزار شد. شرکت در این جشنواره برای همهٔ علاقه‌مندان، آزاد و رایگان بود.

\subsection{روز آزادی نرم‌افزار سال ۱۳۹۴ در تهران}
جشنوارهٔ روز آزادی نرم‌افزار سال ۱۳۹۴ در تهران\LTRfootnote{\lr{\href{http://sfd.fsug.ir/1394}{sfd.fsug.ir/1394}}}، در تاریخ ۹ مهر ۱۳۹۴ مصادف با ۱ اکتبر ۲۰۱۵، از ساعت ۹ تا ۱۷ به وقت تهران در سالن جابرابن‌حیان و دانشکده کامپیوتر \href{http://sfd.fsug.ir/1394/directions}{دانشگاه صنعتی شریف} در دو بخش \href{http://sfd.fsug.ir/1394/plan/presentations}{کنفرانس} و \href{http://sfd.fsug.ir/1394/plan/workshops}{کارگاه‌های آموزشی} برگزار شد.  در طول این مدت بیش از ۸۰۰ نفر از شهرهای مختلف ایران در این جشنواره \href{http://sfd.fsug.ir/1394/about-us/participation}{شرکت کردند}. شرکت در این جشنواره برای همهٔ علاقه‌مندان، آزاد و رایگان بود و بیشتر \href{http://sfd.fsug.ir/1394/patronage/donation}{هزینه‌ها} از طریق دونیشن افراد جامعهٔ نرم‌افزار آزاد در ایران تامین شد.

بیشتر سخنرانی‌های بخش کنفرانس با موضوعات غیرفنی بود. ارائه‌دهندگان بخش کنفرانس دربارهٔ موضوعات \href{https://www.youtube.com/watch?v=gTzkwQ0Qv2Y&index=9&list=PLzkZTZKm4j8GhgZ_O4k7uzJFcA0Ai5FIL}{اهمیت حریم شخصی}، \href{https://www.youtube.com/watch?v=J19dcoRs8So&index=9&list=PLzkZTZKm4j8GhgZ_O4k7uzJFcA0Ai5FIL}{جامعهٔ نرم‌افزار آزاد}، \href{https://www.youtube.com/watch?v=EUx6H9_ah6o&index=8&list=PLzkZTZKm4j8GhgZ_O4k7uzJFcA0Ai5FIL}{معرفی نرم‌افزار آزاد}، معرفی \href{https://www.youtube.com/watch?v=2u899lyBDfM&index=9&list=PLzkZTZKm4j8GhgZ_O4k7uzJFcA0Ai5FIL}{مدارک \lr{LPIC}}، \href{https://www.youtube.com/watch?v=R8vG0poPQv0&index=9&list=PLzkZTZKm4j8GhgZ_O4k7uzJFcA0Ai5FIL}{درس حجیم آنلاین آزاد (ماک)}، \href{https://www.youtube.com/watch?v=CeEOKC9_ctg&index=9&list=PLzkZTZKm4j8GhgZ_O4k7uzJFcA0Ai5FIL}{نرم‌افزار آزاد در کلان‌داده}، \href{https://www.youtube.com/watch?v=3BNNu_hh74k&index=9&list=PLzkZTZKm4j8GhgZ_O4k7uzJFcA0Ai5FIL}{ابزارها، آینده و رهنمودهای ویکی‌پدیا}، \href{https://www.youtube.com/watch?v=cVrVBnRg_Bc&index=9&list=PLzkZTZKm4j8GhgZ_O4k7uzJFcA0Ai5FIL}{بنیاد نرم‌افزارهای آزاد/متن‌باز ایران} و \href{https://www.youtube.com/watch?v=8q8IYF1NH9Q&index=9&list=PLzkZTZKm4j8GhgZ_O4k7uzJFcA0Ai5FIL}{سامانهٔ \lr{DLP}} سخنرانی کردند. ابتدای مراسم فیلم \href{https://www.youtube.com/watch?v=gjhJ6-0kwzc}{پیام ریچارد استالمن} که به مناسبت جشنوارهٔ روز آزادی نرم‌افزار تهران ضبط کرده بود، پخش شد. همچنین در بخش‌‌های استراحت، دو انیمیشن آزاد بنیاد بلندر به نام‌های \lr{\href{https://durian.blender.org/download/}{ Sintel}} و \lr{\href{https://www.youtube.com/watch?v=Y-rmzh0PI3c}{Cosmos Laundromat}}، \href{https://upload.wikimedia.org/wikipedia/commons/0/0c/Experience_ubuntu.ogg}{مصاحبهٔ نلسون ماندلا دربارهٔ اوبونتو} و \href{https://www.youtube.com/watch?v=11VGDAOVEag}{کلیپ \lr{We are Linux}} پخش شد.

در بخش کارگاه‌ها، کارگاه‌های متنوعی با موضوعات فنی و عمومی به صورت موازی با بخش کنفرانس ارائه شد. کارگاه نصب و استفاده از سیستم عامل گنو/لینوکس در لابی دانشکدهٔ کامپیوتر دانشگاه صنعتی شریف از ساعت ۱۰ تا ۱۶ برگزار شد. سایر کارگاه‌ها با موضوعات \href{https://www.youtube.com/watch?v=U_r1nsnIu5c&index=12&list=PLzkZTZKm4j8GhgZ_O4k7uzJFcA0Ai5FIL}{ابزارهای توسعهٔ ویکی‌پدیا}، \href{https://www.youtube.com/watch?v=ORnXkMFw2v4&list=PLzkZTZKm4j8GhgZ_O4k7uzJFcA0Ai5FIL&index=13}{ابزارهای حفظ حریم شخصی}، {\href{https://www.youtube.com/watch?v=BZaZr5K5f64&list=PLzkZTZKm4j8GhgZ_O4k7uzJFcA0Ai5FIL&index=14}{آموزش نرم‌افزار بلندر}، آموزش فریم‌ورک‌های فرانت‌اند، {\href{https://www.youtube.com/watch?v=Qp6fcUIAe-A&list=PLzkZTZKm4j8GhgZ_O4k7uzJFcA0Ai5FIL&index=15}{سخت‌افزار آزاد و آردوئینو}، \lr{\href{https://www.youtube.com/watch?v=tKO6R-N4yy0&list=PLzkZTZKm4j8GhgZ_O4k7uzJFcA0Ai5FIL&index=16}{Embedded Linux From Scratch}}، امنیت (با موضوعات \lr{\href{https://www.youtube.com/watch?v=VakFadk1pEQ&list=PLzkZTZKm4j8GhgZ_O4k7uzJFcA0Ai5FIL&index=17}{SELinux}}، \href{https://www.youtube.com/watch?v=22ijjCLk2dw&list=PLzkZTZKm4j8GhgZ_O4k7uzJFcA0Ai5FIL&index=18}{کالی لینوکس}، \href{https://www.youtube.com/watch?v=XFumj-_CTGQ&list=PLzkZTZKm4j8GhgZ_O4k7uzJFcA0Ai5FIL&index=19}{هک کلاه سفید و تست نفوذ به شبکه})، برنامه‌نویسی (با موضوعات \href{http://sfd.fsug.ir/1394/workshops/73-qt}{نوشتن اپلیکشن با \lr{Qt}} و آشنایی با زبان برنامه‌نویسی \lr{R})، برنامه‌نویسی وب (با موضوعات \href{http://sfd.fsug.ir/1394/workshops/72-yii2}{آموزش فریمورک \lr{YII}}} و کامپوننت‌نویسی برای جوملا) و \href{https://www.youtube.com/watch?v=BvU-Z7oXvmg&list=PLzkZTZKm4j8GhgZ_O4k7uzJFcA0Ai5FIL&index=20}{سامانهٔ کنترل ورژن گیت} در کلاس‌های دانشکدهٔ کامپیوتر برگزار شد. همچنین از مجموع ۲۳۲ پیامک ارسال شده در مسابقهٔ پیامکی ارائه اول با ۳۴ پیامک، ارائه دوم با ۳۳ پیامک و ارائه سوم با ۳۲ پیامک برنده شدند و جوایزی از طرف اسپانسر جشنواره به  \href{https://twitter.com/tehsfd/status/725078247520788482}{این افراد} تقدیم شد.

همچنین یکی از بخش‌های موازی در این جشنواره \href{http://sfd.fsug.ir/1394/paint}{نقاشی لوگوهای نرم‌افزار آزاد روی تیشرت} بود. در این بخش افراد تیشرت‌های خود را در اختیار برگزارکنندهٔ آن قرار دادند و او روی لباس‌های افراد طرح‌های مورد نظر را نقاشی کرد که سود حاصل از آن صرف هزینه‌های خود جشنواره شد.

در این جشنواره ۱۶ نفر به عنوان \href{http://sfd.fsug.ir/1394/patronage/staff}{تیم اجرایی} و برگزارکنندهٔ رویداد و ۱۹ نفر به عنوان استف در روز جشنواره مشارکت کردند و از میان ۳۳ نفری که در سایت جشنواره \href{http://sfd.fsug.ir/1394/plan/articles}{مقاله یا پیشنهاد برگزاری کارگاه} ثبت کردند بعد از برگزاری \href{http://sfd.fsug.ir/1394/referee}{جلسات داوری} ۱۳ نفر در بخش کارگاه‌ها و ۱۰ نفر در بخش همایش مطالب خود را ارائه کردند و شش عدد کلیپ جمعا به مدت ۴۸ دقیقه در سالن اصلی پخش شد. متاسفانه به دلیل نواقص فنی بعضی از ارائه‌ها یا کارگاه‌ها ضبط نشدند یا نیمه‌کاره ضبط شدند با این حال از مجموع ۲۴ ساعت محتوای ارائه شده در بخش کارگاه‌ها و همایش، بعد از تدوین جمعا ۱۲ ساعت فیلم سخنرانی به دست آمد که هر کدام در پنج قالب مختلف در سایت‌های \href{https://www.youtube.com/channel/UCbhw92WI8GJZDdBMXn4ONXA?sub_confirmation=1}{یوتیوب}، \href{http://takhtesefid.org/user/freeknowledgefoundation}{تخته‌سفید} و آرشیو آپلود شدند.

میزان \href{http://sfd.fsug.ir/1394/patronage/donation}{حمایت‌های مالی} انجام شده مبلغ یک میلیون و چهارصد و سی هزار تومان توسط پنجاه نفر و با بالاترین مبلغ پانصد هزار تومان بوده است. در این رویداد در مجموع حدود دو میلیون و پانصد هزار تومان خرج شده که گزارش ریز آن در سایت جشنواره منتشر شده است.

\href{http://sfd.fsug.ir/1394/in-media}{بازتاب جشنواره در رسانه‌ها} به گونه‌ای بود که تعدادی از نشریات، فروم‌های اینترنتی، بلاگ‌ها، سایت‌های خبری، سایت‌های مرتبط با نرم‌افزار آزاد و شرکت‌های فناوری اطلاعات، خبر برگزاری جشنواره را منتشر کردند. همچنین شبکهٔ تلویزیونی الکوثر \href{https://www.youtube.com/watch?v=pT-4Gc1Z4LQ}{گزارشی به زبان عربی از جشنواره} تهیه و پخش کرد. \href{http://sfd.fsug.ir/1394/artworks}{طرح‌های گرافیکی} استفاده شده در این گزارش‌ها و سایت‌ها از طریق سایت رسمی جشنواره منتشر شدند.

در انتها، پس از تقدیر از تیم‌اجرایی، ارائه‌دهنده‌ها و حامیان مالی جشنواره، با \href{http://sfd.fsug.ir/1394/photos}{عکس دسته‌جمعی} شرکت‌کنندگان، مراسم به پایان رسید.

\section{اهداف کلی همایش}
سعی بر آن شده تا با برگزاری ارائه‌ها و کارگاه‌های متنوع و با بهره‌گیری از تجارب افراد فعال در جامعهٔ نرم‌افزار آزاد و دانش آزاد، قدم‌هایی در راستای آشنایی بیشتر افراد با این مفاهیم و گسترش آن در جامعه برداشته شود و فرهنگ مشارکت جمعی افزایش پیدا کند. به طور خلاصه می‌توان گفت محورهای اصلی برنامه‌های این جشنواره معرفی، ترویج، افزایش کاربرد و استفاده، جذب مشارکت افراد و جلب حمایت جامعه و تولید در زمینه‌ی نرم‌افزار آزاد، دانش آزاد و سخت‌افزار/روباتیک آزاد است. به طور کلی اهداف همایش به دو صورت دسته‌بندی می‌شود.

\subsection{اهداف موضوعی}
\begin{flushright}

\begin{itemize}
\item نرم‌افزار آزاد
\item سخت‌افزار/رباتیک آزاد
\item دانش آزاد
\end{itemize}
\end{flushright}

\subsection{اهداف اجرایی}

\begin{flushright}
\begin{itemize}
\item ترویج
\item کاربرد و استفاده
\item معرفی
\item جذب مشارکت‌کننده و حمایت جامعه
\item تولید
\end{itemize}
\end{flushright}

به ازای هر کدام از اهداف موضوعی پنج هدف اجرایی تعریف شده‌است. برای نمونه نرم‌افزار آزاد در اجرا پنج ما به‌ازا دارد: ترویج نرم‌افزار آزاد، کاربرد و استفاده از نرم‌افزار آزاد، معرفی نرم‌افزار آزاد، جذب مشارکت‌کننده و حمایت جامعه از نرم‌افزار آزاد و پروژه‌های مرتبط و تولید نرم‌افزار آزاد. برای بقیه اهداف موضوعی مانند دانش آزاد و سخت‌افزار/رباتیک آزاد نیز اهداف اجرایی به این صورت تعیین شده‌است.

\section{منافع حمایت از جامعه}
مخاطبان این رویداد از هر گروه و دسته‌ای مانند توسعه‌دهندگان نرم‌افزار، مدیران سیستم‌ها و شبکه‌های کامپیوتری، دانشجویان، مدیران صنایع و افراد علاقه‌مند به حوزهٔ IT  هستند. به طور کلی مخاطبین این جشنواره افراد فعال و نخبه یا علاقه‌مند در زمینهٔ نرم‌افزار آزاد، شبکه، امنیت و اینترنت هستند.

\begin{flushright}

باتوجه به ماهیت نرم‌افزار آزاد حضور شما به عنوان حامی در این رویداد می‌تواند علاوه‌ بر تبلیغات برند شما، یک حرکت اخلاقی و معنوی در راستای گسترش سطح دانش و رفاه جامعه باشد. به طور کلی نتایجی که حمایت از جامعه در پی دارد به شرح زیر است:

\begin{itemize}
\item حمایت از یک حرکت اخلاقی و معنوی
\item گسترش سطح دانش و رفاه جامعه
\item معرفی فعالیت‌ها، تاریخچه و برند شرکت
\item شناسایی افراد فعال و نخبه و جذب آن‌ها
\item آشنایی و ارتباط با اعضای بقیه گروه‌های کاربری فعال در حوزهٔ نرم‌افزار آزاد و دانش آزاد
\item ارتباط با مدیران و صاحبان صنایع مرتبط با نرم‌افزار آزاد/متن‌باز و افراد فعال و دارای دانش کافی در این زمینه
\end{itemize}

\end{flushright}

\subsection{مواردی که می‌توانید به عهده بگیرید}


\begin{flushright}

\begin{itemize}
\item سالن کنفرانس و کلاس برای کارگاه‌ها
\item ناهار
\item پذیرایی
\item اینترنت
\item پخش زنده
\item تبلیغات کاغذی
\end{itemize}

\end{flushright}

\begin{flushright}

ویژگی‌های مناسب برای مکان برگزاری:

\begin{itemize}
\item عدم نیاز به ثبت نام برای ورود به محل برگزاری رویداد
\item اندازهٔ کافی سالن همایش و سایر فضاها
\item تناسب امکانات سالن با موضوع همایش
\item ارائهٔ امکانات بیشتر از سالن مثل استند برای بنر یا اینترنت و ... 
\end{itemize}

\end{flushright}

\subsection{مواردی که می‌توانیم به عهده بگیریم}

\subsubsection{امتیازات ویژه}

\begin{flushright}
\begin{itemize}
\item قرارگیری و توزیع بروشور در بسته برنامه
\item قرارگیری فرم جذب نیروی کار در سایت برای مدت معلوم
\item اختصاص یک عدد میز برای توزیع بروشور، فرم جذب نیروی کار و معرفی شفاهی شرکت در فضای پذیرایی بیرون سالن
\item درج نشان شرکت روی پوسترهای جشنواره و بنرهای اینترنتی
\item قرارگیری پرچم حامی در سن برنامه
\item اهدای تندیس و لوح تقدیر از حامی از طرف دبیرخانه جشنواره
\item تقدیر از حامی در پایان برگزاری مراسم
\end{itemize}
\end{flushright}

\subsubsection{امتیازات معمولی}
\begin{flushright}

\begin{itemize}
\item ثبت نام و نشانه و درج لینک شرکت در بخش حامیان ویژه در سایت جشنواره
\item نصب استند در بیرون سالن
\item قرار گیری و توزیع فرم جذب نیروی کار در بسته برنامه
\item درج نام شرکت در تاریخچهٔ سایت برای همیشه
\item معرفی حامی در شبکه‌های اجتماعی مربوط به رویداد
\item درج لوگو و نام حامی در ایمیل‌های گروهی
\item معرفی حامی در اطلاع‌رسانی‌ها و اخبار ارائه‌شده توسط حامیان رسانه‌ای
\end{itemize}
\end{flushright}

\end{document}



